\documentclass{flownotes}
  
\begin{document}
% notes
\begin{notebox}{Doc\#19072501}
\tcbsubtitle{A digital template for my notebook}
At some point in July, 2019,
I got the idea of maybe  developing a digital version of my note taking method,
using \XeLaTeX.
I hope it could be easy and efficient to use and the same time it keeps all the features of my original on-paper note taking method.
If this is possible, then I can get rid of a heavy notebook being carried all the time and yet I can still take notes like I have been doing.
\tcbline
The template is based on the \noteurl[tufte-book]{https://github.com/Tufte-LaTeX/tufte-latex} class.
\tcbsubtitle{font settings and usage}
{\centering
\resizebox{\columnwidth}{!}{
\begin{tabular}{cccc}
                \toprule
                \multicolumn{1}{c}{\textbf{Language}} &
                \multicolumn{1}{c}{\textbf{Style}} &
                \multicolumn{1}{c}{\textbf{Font}} &
                \multicolumn{1}{c}{\textbf{Examples}}  \\
                \midrule

                \begin{tabular}{l}
                    Latin\\
                \end{tabular}
                &
                \begin{tabular}{l}
                    Serif\\
                    Sans Serif\\
                    Monospace\\
                \end{tabular}
                &
                \begin{tabular}{r}
                    Bembo\\
                    GillSans\\
                    Menlo\\
                \end{tabular}
                &
                \begin{tabular}{r}
                    ABC123\\
                    \textsf{ABC123}\\
                    \texttt{ABC123}\\
                \end{tabular}
                \\
                \midrule
                \begin{tabular}{l}
                    CJK\\
                \end{tabular}
                &
                \begin{tabular}{l}
                    Serif\\
                    Sans Serif\\
                    Monospace\\
                \end{tabular}
                &
                \begin{tabular}{r}
                    SourceHanSerif\\
                    SourceHanSans\\
                    SourceHanMono\\
                \end{tabular}
                &
                \begin{tabular}{r}
                    汉字かな한글\\
                    {\sans 汉字かな한글}\\
                    {\sans 汉字かな한글}\\
                \end{tabular}
                \\
                \bottomrule
\end{tabular}}\par
}
\tcbline
The default CJK glyphs are for Simplified Chinese. To change them for a specific CJK language, use the following commands:
\begin{lstlisting}[language=TeX,
                  basicstyle=\coding\jp,
                  commentstyle=\color{gray},
                  keywordstyle=\color{blue}\bfseries]
    \jp  % Japanese: 骨文才免返曜
\end{lstlisting}
\begin{lstlisting}[language=TeX,
                  basicstyle=\coding\kr,
                  commentstyle=\color{gray},
                  keywordstyle=\color{blue}\bfseries]
    \kr  % Korean: 骨文才免返曜
\end{lstlisting}
\begin{lstlisting}[language=TeX,
                  basicstyle=\coding\sch,
                  commentstyle=\color{gray},
                  keywordstyle=\color{blue}\bfseries]
    \sch  % Simplified Chinese: 骨文才免返曜
\end{lstlisting}
\begin{lstlisting}[language=TeX,
                  basicstyle=\coding\tch,
                  commentstyle=\color{gray},
                  keywordstyle=\color{blue}\bfseries]
    \tch  % Traditional Chinese: 骨文才免返曜
\end{lstlisting}
% \makeatletter
% \@for\sun:={rising,setting}\do{The sun is \sun. }
% \makeatother
\tcbline
other useful commands:
\begin{lstlisting}[language=TeX,
                  basicstyle=\coding,
                  commentstyle=\color{gray},
                  keywordstyle=\color{blue}\bfseries]
    \coding  % Menlo + Source Han Mono
    \sansall % Source Han Sans for all chars
    \sans % GillSans + Source Han Sans
\end{lstlisting}

\end{notebox}

\clearpage
\rfnote[0.8cm]{2}{blue}{Example of referring to text in a note box}
\rfnote[0.8cm]{4}{red}{Cascaded ref note}
\rfnote[0.8cm]{3}{red}{\markinmargin{4}{purple}{Example}of referring to text in the margin}
\begin{notebox}{Doc\#19072801}
\tcbsubtitle{Continue on Doc\#19072501 -- reference notes}
Drawing arrows to refer to some text is a very frequent in my notes.\markintext{2}{blue}{This example}links to the reference note on the margin of this page. It is done by:
{\lstset{language=[LaTeX]{TeX},alsoletter={\\},emph={\\,/,notebox},emphstyle=\color{lime!60!black},alsoletter={/}}
\begin{lstlisting}[
                  basicstyle=\coding,
                  commentstyle=\color{gray},
                  frameround=fttt,
                  frame=trBL,
                  morekeywords={rfnote},
                  morekeywords={markintext},
                  keywordstyle=\color{blue}\bfseries]
 % \clearpage
   \rfnote[<offset>]{<ref id>}{<color>}{<text>}
   \begin{notebox}/
     \markintext{<red id>}{<color>}{<text>}
   \end{notebox}
\end{lstlisting}}
\texttt{<offset>} means to adjust the end point of the arrow on the ref note. We need this because there is some unknown bug on getting the exact position of a tikz node on the margin. However, this is optional. It should be something like \texttt{0.8cm}. \texttt{<offset>} must be an integer. \texttt{<color>} can be \texttt{blue}, \texttt{red}, etc. The two \texttt{<text>}s are linked texts.
\tcbline
I also want to allow cascading as well.\markintext{3}{red}{This is an example.}It is done by:
\begin{lstlisting}[language=TeX,
                  basicstyle=\coding,
                  commentstyle=\color{gray}]
  % \clearpage
  \rfnote[<offset>]{<ref id #1>}{<color>}{<text>}
  \rfnote[<offset>]{<ref id #2>}{<color>}{
     \markinmargin{<ref id #1>}{<color>}{<text>}}
  \begin{notebox}{<title>}
     \markintext{<ref id #2>}{<color>}{<text>}
  \end{notebox}
\end{lstlisting}
Note that when the linked \texttt{<text>} is in the margin, then you should use \texttt{\textbackslash markinmargin} instead of \texttt{\textbackslash markintext}.

I recommend using the same color for a whole cascading chain so that it is clear to see.
\tcbsubtitle{To structure a note}
The main part of a note is located in a note box environment. we can use blue-background subtitles and dash lines to structure a note. The related commands are:
\begin{lstlisting}[language=TeX,
                  basicstyle=\coding,
                  commentstyle=\color{gray}]
  \begin{notebox}{<title>}
     \tcbsubtitle{<subtitle>}
     \tcbline  % dash line
  \end{notebox}
\end{lstlisting}
\end{notebox}


\clearpage
\rfnote[0.8cm]{5}{blue}{See here!}
\begin{notebox}{Doc\#19072901}
\tcbsubtitle{vertical texts}
Since CJK texts can be formatted vertically, I think it is beneficial to have a simple command to do that.
\cjkvert[\sans]{0.45\textwidth}{
{\sch 这,是一句中文竖排字符。}\\
\ \\
{\jp これは、\ruby{縦書}{た\ \ て\ \ が}きの文字です。}\\
\ \\
不建议用半角字符ABC123\\
\ \\
建议用全角的ABC123\\
\ \\
{\jp \symbol{"32FF}元年七月二十九日。}\\
}
To do this, use:
\begin{lstlisting}[language=TeX,
                  basicstyle=\coding,
                  commentstyle=\color{gray}]
\cjkvert[<commands>]{<vertical length>}{<text>}
\end{lstlisting}
For example, to get the above output, \texttt{<commands>} should be \texttt{\textbackslash sans} and \texttt{<vertical length>} should be \texttt{0.45\textbackslash textwidth}
\tcbline
I aslo want to use Furigana when I am writting some Japanese texts. You can do this as the following example:
\begin{lstlisting}[language=TeX,
                  basicstyle=\coding\jp,
                  commentstyle=\color{gray}]
  \ruby{外貨}{がいか}預金には、
  \ruby{為替}{かわせ}差益や為替差損があります。
\end{lstlisting}
The result is:

{\centering\jp \ruby{外貨}{が\ \ い\ \ か}預金には、
\ruby{為替}{か\ \ わ\ \  せ}差益や為替差損があります。\par}
\end{notebox}
\begin{fullwidth}
\begin{notebox}{Doc\#19073001}
\tcbsubtitle{expand the note box to the whole page}
We can do this using the \texttt{fullwidth} environment provided by \noteurl[tufte-book]{https://github.com/Tufte-LaTeX/tufte-latex} package itself outside the \texttt{notebox} environment.\markintext{5}{blue}{ref note should still work.}However, I don't recommend expanding the note box when using ref notes because some ref notes maybe hidden by the note box.

\end{notebox}
\end{fullwidth}
\clearpage
\begin{notebox}{Doc\#19080201}
\tcbsubtitle{Q\&A}
Q\&A is a very effective way of taking notes. It helps you to get the point of a note very quickly because knowledge is essentially about answering questions. If there are no questions, there is no knowledge.
\qa[2]{The question goes here}{The answer goes here}
To do this, use the following command:
\begin{lstlisting}[language=TeX,
                  basicstyle=\coding,
                  commentstyle=\color{gray}]
    \qa[<q&a id>]{<question>}{<answer>}
\end{lstlisting}
\end{notebox}
\begin{notebox}{Doc\#190100101}
\tcbsubtitle{Updates on the template}
We now make \textbf{Bold} and \textit{Italic} fonts possible. Here is an example list for both Latin and CJK characters.

{\centering
\begin{tabular}{cccc}
                \toprule
                \multicolumn{1}{c}{\textbf{Language}} &
                \multicolumn{1}{c}{\textbf{Style}} &
                \multicolumn{1}{c}{\textbf{Bold}} &
                \multicolumn{1}{c}{\textbf{Italic}}  \\
                \midrule

                \begin{tabular}{l}
                    Latin\\
                \end{tabular}
                &
                \begin{tabular}{l}
                    Serif\\
                    Sans Serif\\
                    Monospace\\
                \end{tabular}
                &
                \begin{tabular}{r}
                    \textbf{ABC123}\\
                    \textbf{\textsf{ABC123}}\\
                    \textbf{\texttt{ABC123}}\\
                \end{tabular}
                &
                \begin{tabular}{r}
                    \textit{ABC123}\\
                    \textit{\textsf{ABC123}}\\
                    \textit{\texttt{ABC123}}\\
                \end{tabular}
                \\
                \midrule
                \begin{tabular}{l}
                    CJK\\
                \end{tabular}
                &
                \begin{tabular}{l}
                    Serif\\
                    Sans Serif\\
                    Monospace\\
                \end{tabular}
                &
                \begin{tabular}{r}
                    \textbf{汉字かな한글}\\
                    \textbf{\sans 汉字かな한글}\\
                    \textbf{\coding 汉字かな한글}\\
                \end{tabular}
                &
                \begin{tabular}{r}
                    \textit{汉字かな한글}\\
                    \textit{\sans 汉字かな한글}\\
                    \textit{\coding 汉字かな한글}\\
                \end{tabular}
                \\
                \bottomrule
\end{tabular}\par
}
\tcbline
We also make a new design on \textit{concept} element.\\
We use a \textit{name-meaning-example} triplet to explain a concept. \\
\concept{name}{meaning}{examples}
To do this, use the following command:
\begin{latexblock}{}{concept}{\small}
    \concept[<name>]{<meaning>}{<examples>}
\end{latexblock}
\tcbline
We use curly brackets to group concepts.
\begin{itemize}
\item A
\item B
  \begin{fork}
  \item B1
    \begin{fork}
      \item B11
      \item B12
    \end{fork}
  \item B2\tikzmark{a}
  \item B3
  \item B4\tikzmark{b}
  \end{fork}
\item C
\end{itemize}
\merge{a}{b}[merge to $\text{B}^\prime$]
To use forking brackets, make sure you are inside an itemize or enumerate environment, and then use the following envronment:
\begin{latexblock}{fork}{}{\small}
    \item A
    \begin{fork}
      \item A1
      \item A2
    \end{fork}
\end{latexblock}
To use merging brackets, mark the two endpoints with {\coding \textbackslash tikzmark\{name1\}} and {\coding \textbackslash tikzmark\{name2\}}, and then use the following commands after them:
\begin{latexblock}{}{merge}{\small}
    \merge{name1}{name2}[concept C]
\end{latexblock}
\tcbline
Another update is that we now make listing latex code more easy with the following environment (delete the space in {\coding latexbloc\ k}):
\begin{latexblock}{}{merge}{\footnotesize}
  \begin{latexbloc k}{emph1,...}{kw1,...}{style commands}
    \keyword{emphasized word}
  \end{latexbloc k}
\end{latexblock}
\end{notebox}
\clearpage
\begin{fullwidth}
\begin{notebox}{Doc\#19100201}
\tcbsubtitle{updates on text emphasis}
{\centering
\begin{tabular}{cc}
                \toprule
                \multicolumn{1}{c}{\textbf{Command}} &
                \multicolumn{1}{c}{\textbf{Effect}} \\
                \midrule

                \begin{tabular}{l}
                    {\coding\textbackslash strikethrough[color]\{text\}}\\
                    {\coding\textbackslash crossout[color]\{text\}}\\
                    {\coding\textbackslash highlight[color]\{text\}}\\
                    {\coding\textbackslash
                    underline[color]\{text\}}\\
                    {\coding\textbackslash underdbline[color]\{text\}}\\
                    {\coding\textbackslash underwave[color]\{text\}}\\
                    {\coding\textbackslash underdot[color]\{text\}}\\
                    {\coding\textbackslash undertriangle[color]\{symbol\}\{text\}}\\
                    {\coding\textbackslash undercircle[color]\{symbol\}\{text\}}\\
                    {\coding\textbackslash undercross[color]\{symbol\}\{text\}}\\
                \end{tabular}
                &
                \begin{tabular}{l}
                    \strikethrough[red]{AaBbCc123汉字かな한글}\ \\
                    \crossout[red]{AaBbCc123汉字かな한글}\\
                    \highlight{AaBbCc123汉字かな한글}\\
                    \underline[red]{AaBbCc123汉字かな한글}\\
                    \underdbline[red]{AaBbCc123汉字かな한글}\\
                    \underwave[red]{AaBbCc123汉字かな한글}\\
                    \underdot[red]{AaBbCc123汉字かな한글}\\
                    \undertriangle[red]{AaBbCc123汉字かな한글}\\
                    \undercircle[red]{AaBbCc123汉字かな한글}\\
                    \undercross[red]{AaBbCc123汉字かな한글}\\

                \end{tabular}
                \\
                \bottomrule
\end{tabular}\par
}
\end{notebox}

\end{fullwidth}
\clearpage
\bibliography{docbib}
\bibliographystyle{abbrv}
\end{document}

